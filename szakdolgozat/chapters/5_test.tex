\Chapter{Az elkészült függvénykönyvtár bemutatása}

\Section{Random keresés}

\subsection{Példa útvonal generáló python fájl}

A $random\_search/example\_path\_generator.py$ fájl függvényeinek használatát szeretném bemutatni, ami kísérleti jelleggel készült. A paraméterek nagyban meghatározzák a generált útvonal hosszát, kezdő és végpontját, valamint a kanyarok élességét. 

Az $is\_in\_wall(x, y)$ függvényt az ütközésdetektáláshoz használja a program, ezt nem szükséges paraméterezni. Visszaadja viszont, hogy volt-e ütközés, az objektum indexét, valamint a koordinátákat ahol az ütközés történt. \textbf{A fejezet további részeiben a függvényekből csak a paraméterlistát és a visszatérési értéke(ke)t, vagy egyéb olyan változókat fogok bemutatni, amik segítségével módosítható a kimenet.}

\begin{python}
def is_in_wall(x, y):

	return collided, collisions
\end{python}

A $ generate\_random\_angles\_absolute(number\_of\_turns) $ és a\\ $ generate\_random\_angles\_relative(number\_of\_turns) $ függvényeket meghívva generálhatóak az elfordulási szögek, a paraméter változtatásával az út hosszát tudjuk módosítani, a törzsön belül szereplő $ angle $ változóval pedig a kanyarodások élességét.

\begin{python}
def generate_random_angles_relative(number_of_turns):

	for i in range(number_of_turns):
        	angle = random.randint(-25, 25)
        
    return rel_angles
\end{python}

A $ calc\_path(pos\_x, pos\_y, steering\_angles) $ függvény útvonal számítást végez és ütközést vizsgál. Hívásakor a kezdőpozíciót kell megadni az első két paraméterben, valamint a generált szögek listáját. Az első visszatérési értéke az útvonal, a második az ütközések koordinátái.

\begin{python}
def calc_path(pos_x, pos_y, steering_angles):

	return path, collisions
\end{python}

A visszatérési értékeket pedig változókban eltárolva, az eredményt a plot() függvény segítségével meg lehet jeleníteni. 

\subsection{Hisztogram készítés}

A $ random\_search/x\_y\_endpoints\_histogram.py $ és az\\ $ x\_y\_endpoints\_histogram.py $ fájlokat használhatjuk a 2 dimenziós hisztogramok készítésére, mely az előbbiekben bemutatott függvényekből áll, valamint egy for ciklusból, ami a generálás mennyiségét adja meg. Ezt a következő helyen tudjuk paraméterezni, ahol jelenleg 10000 esetet generál futtatáskor:

\begin{python}
for i in range(10000):
    def generate_random_angles_relative(number_of_turns):
\end{python}

\Section {A* keresés}

\subsection{Az útvonal keresés python fájl}

Az A* keresést megvalósító fájl a $ navigator/pathfinder.py $ lenne. Ezene belül található Az $ AStarPathFinder $ osztály, amit inicializálunk akadályokkal az $\_\_init\_\_(self)$ függvényen belül. Itt lehet megadadni még plusz akadályokat, vagy átszervezni azokat. A többi függvényt az algoritmus fogja használni, amiket már korábban kifejtettem és megnéztük a működésüket. Ahhoz, hogy ez a kereső függvény használni tudja ezeket a metódusokat, paraméterként kell átadni ennek az osztálynak a példányát és erre fog hivatkozni a belső számításoknál. A hivatkozások a következő példához hasonlóan történnek:

\begin{python}
def a_star_search(start, end, map):

	map.check_points(start, end)
\end{python}

A következő lépés itt a $ main $ függvény, ahol a példányosításnak történnie kell. Majd meghatározzuk a másik két paramétert is, ami a kezdő és végpozíciók koordinátáit jelenti. A kereső függvény, ami az algoritmust valósítja meg az osztály függvényeit használva az útvonallal és annak költségével tér vissza. Ennek az eredményét szintén a plot() függvénnyel nézhetjük meg legegyszerűbben.

\begin{python}
if __name__ == "__main__":
    map = AStarPathFinder()

    start = (0, 0)
    goal = (20, 3)

    path, cost = a_star_search(start, goal, map)
    
    x_coords = list(zip(*path))[0]
    y_coords = list(zip(*path))[1]
    
    plt.plot(x_coords, y_coords, 'black')
    plt.plot(start[0], start[1], 'ro')
    plt.plot(goal[0], goal[1], 'go')
\end{python}

\Section {A jármű mozgása}

\subsection{A mozgatáshoz használt python fájl}

A mozgatáshoz a $ positioning/moving\_vehicle $ fájlt kell futtatni. Ennek paraméterezhetősége kezdetleges, változókkal tudjuk a mozgás ívét meghatározni. Itt arra kell figyelni leginkább, hogy a $ k\_alpha $ pozitív szám legyen, valamint a $ k\_beta $ negatív. Példányosításra kerül a Parkoló osztály, aminek a kirajzoló metódusát később hívjuk meg. Valamint a jármű mérete állítható. 

\begin{python}
k_distance = 2
k_alpha = 6
k_beta = -4
dt = 0.01

draw = parking_lot.ParkingLot()

p1_init = np.array([1, -0.5, 1])
p2_init = np.array([1, 0.5, 1])
p3_init = np.array([-1, 0.5, 1])
p4_init = np.array([-1, -0.5, 1])
\end{python}

A $ plot\_vehicle(x, y, heading\_angle, x\_path, y\_path, steps) $ függvény játszik szerepet a kirajzolásban és az animáció készítésben,  mivel paraméterként mindent megkap a jármű mozgásáról, valamint a lépésekről. Ez a képek generálásához nyújt segítséget, ugyanis növekményes sorszámmal látja el őket. A rotációs mátrix is a törzsön belül kerül meghívásra, ahogy a parkoló kirajzolása is. A $ moving\_vehicle(x\_start, y\_start,$ \\ $start\_heading\_angle, x\_goal, y\_goal, goal\_heading\_angle) $ metódus kapja meg a kezdő és végpozíció paramétereit és végzi el a számításokat. 

\begin{python}
def plot_vehicle(x, y, heading_angle, x_path, y_path, steps):

	r = rotation_matrix(x, y, heading_angle)
	
	draw.plot_parking_lot()
	
	plt.plot(x_path, y_path, 'b--')
    	plt.savefig('pictures/'f'fig-0{steps}.png')
\end{python}

A $ moving\_vehicle $ függvény törzsében még fontos szerepet játszik a tolatás feltétele, ennek az értékét is érdemes változtatni kisebbre, ha egy parkoló helyről való kitolatásról van szó. Az ehhez tartozó animációnál a feltétel a következő volt:

\begin{python}
if alpha > pi/3 or alpha < -pi/3:
      v = -9
\end{python}

Itt is a $ main $ függvényben adhatjuk meg a paraméterek értékeit:

\begin{python}
if __name__ == '__main__':
    x_start = 6.5
    y_start = 8
    start_heading_angle = pi/2
    x_goal = 18.5
    y_goal = 9
    goal_heading_angle = -pi/2

    moving_vehicle(x_start, y_start, start_heading_angle, x_goal, y_goal,
     goal_heading_angle)
\end{python}


