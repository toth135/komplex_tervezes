
\Chapter{Összefoglalás}

Dolgozatom a négykerekű járművek alapvető navigációs problémáit elemzi, melynek alapja az útvonalkeresés. Példákon keresztül láthattunk eredményeket különböző véletlen bolyongásos kísérletekre, amelyeket összesítettem és kiértékeltem az adott paraméterek alapján. Az A* algoritmus, mint hatékony keresési eljárás segítségével megtaláltuk a legrövidebb utat két pont között egy olyan környezetben,amit a benne található akadályokkal mi magunk alakíthatunk ki és bármilyen távolságba eljuthatunk. Majd animációkkal szemléltettem a pozícionálás problémáit haladás közben. Itt szerettem volna megoldani még az ütközésdetektálás és elkerülés prolémáját, ami sajnos továbbfejlesztési lehetőségnek maradt meg. Ennél a példánál teljesen máshogy térképezi fel a bejárandó pontokat a program, mint egy gráfszerű keresés során. Így egy üres parkolót láthatunk, a terveimben pedig parkoló autók is szerepeltek, amiket jó lett volna kikerülni a mozgás közben.

Természetesen ezeken felül még további problémák is felmerülhetnek egy jármű navigálása közben. Például definiálni lehetne többféle algoritmust az útvonal meghatározására (kevesebb kanyarodással bejárható út, legkevesebb idő alatt megtehető út). Vagy olyan helyzetek kialakítása, ahol a jármű mérete is számít és emiatt újra kell terveznie az utat, vagy épp korrigálnia kell hátramenetbe kapcsolva.

