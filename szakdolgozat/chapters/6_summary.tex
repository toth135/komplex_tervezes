\Chapter{Összefoglalás}

A dolgozatom a négykerekű járművek alapvető navigációs problémáit elemzi, mely leginkább az útvonal keresési technikákra tért ki. Végeredményként pedig az útvonalon lévő, valamint kezdő és végpontokban fennálló pozíciónálási feladatokat oldja meg. Példákon keresztül láthattunk eredményeket a véletlenszerű útvonal keresésre, amelyeket összesítettem és kiértékeltem az adott paraméterek alapján. Az A* algoritmus, mint hatékony keresési eljárás segítségével megtaláltuk a legrövidebb utat két pont között egy olyan környezetben, ahol bármennyi akadályt le lehet helyezni és bármilyen távolságba eljuthatunk. Majd animációkkal szemléltettem a pozícionálás problémáit haladás közben. Ennél a résznél viszont, amit már nem sikerült megvalósítani, az az ütközés detektálás, ugyanis itt teljesen máshogy térképezi fel a bejárandó pontokat a program, mint egy keresőeljárás során. Így egy üres parkolót láthatunk, a terveimben pedig parkoló autók is szerepeltek, amiket jó lett volna kikerülni a mozgás közben.

Természetesen ezeken felül még további problémák is felmerülhetnek egy jármű navigálása közben, például többféle algoritmus az útvonal meghatározására (kevesebb kanyarodással bejárható út, legrövidebb alatt megtehető út). Vagy olyan helyzetek kialakítása, ahol a jármű mérete is számít, mondjuk két szűk akadály között emiatt nem tud elmenni és vissza kell tolatnia. Ezek mind továbbfejlesztési lehetőségek a program szempontjából.